\documentclass[envcountsame]{llncs}%

%---------------------
%---------------------
%---------------------
%\pagestyle{plain}

\title{Partial-Order Subsumption}

\usepackage{comment}
\usepackage{amsmath}
\usepackage{amssymb}
\usepackage{graphicx}
\usepackage{array}
\usepackage{listings}


\newcommand{\apost}{\mathit{apost}}
\newcommand{\labeling}{\rho}
\author{Bj\"orn Wachter}
\institute{Department of Computer Science, Oxford}

% switch between long and short version
\def\version{short}

\def\longversion{long}
\def\shortversion{short}

\sloppy

\begin{document}

\maketitle

\begin{abstract}
  Partial-order reduction is a systematic technique
  to tackle the interleavings explosion in verification
  of concurrent systems.
  %
  Partial-order reduction admits a concise automata-theoretic
  characterisation, such that a concurrent system
  can be verified by forming the cross product
  of the original system with a partial-order automaton,
  that filters out redundant execution traces.
  %
  However, so far, this approach has only been explored
  without a notion of convergence, which is required
  to compute loop invariants.
  %
  Subsumption provides an effective means to detect
  convergence. 
  %
  In this paper, we present the missing piece, a subsumption
  order for partial-order automata, and use it to
  compute loop invariants for concurrent systems.
\end{abstract}


\section{Background}

\subsection{Automata}

\begin{definition}[Automaton]
  $(Q,\Sigma,\delta,q_0,F)$
  \begin{itemize}
    \item states $Q$,
    \item alphabet $\Sigma$,
    \item transition function $\delta:Q\times \Sigma \to (Q \cup \{\epsilon\})$
    where $\epsilon$ is a special sink state,
    \item initial states $q_0\in Q$,
    \item final states $F\subseteq Q$.
  \end{itemize}
\end{definition}

For a state $q\in Q$, we denote by $succ(q)$ the \emph{successor set}:
\begin{align*}
  succ(q):=\{r\in Q_\epsilon \mid r\neq \epsilon \wedge \exists \sigma \in \Sigma:\ r=\delta(q,\sigma)\} \ .
\end{align*}

\begin{definition}[Traces of an Automaton]
  
\end{definition}

\begin{definition}[Simulation Relation]
  Let $A=(Q,\Sigma,\delta,q_0,F)$ be an automaton.
  A simulation relation $R\subseteq Q\times Q$ is a binary relation
  such that:
\begin{align*}
  (q,q')\in R \ \Longrightarrow \ \forall r\in succ(q)\ \exists r'\in succ(q'): (r,r')\in R \ .
\end{align*}
\end{definition}

\subsection{Dependency-Chain Automata}

In this subsection, we revisit an automata-theoretic characterisation
of Mazurkievicz traces, in terms of the language of an automaton.
% FIXME: need to say what M. traces are beforehand.
%
We develop this automaton in two steps.
As a first step, we give an automaton that only observes
shared accesses and stores resulting transitive dependencies
between threads.
The language of this automaton is the set of all possible interleavings.
%
In a second step, we define a constrained automaton that 
filters out redundant, i.e., Mazurkievicz-equivalent traces.
%
\begin{definition}[Dependency-Chain Automaton]

% - 
% - dependencies between variable accesses
% - a pair of accesses from two different threads
%   (without a happens-before relationship) to the same memory location 
%   creates a conflict, in that one of the accesses must come first in any given trace; 
%   the different ordering within a trace result in different states.
% - 
%

The dependency-chain automaton $A_D$ is defined as follows:
%
\begin{itemize}

  \item states $Q_D= ({2^{Access}})^T \times \{-1,0,1\}^{T\times T}$
  \begin{itemize}
    \item set of accesses per thread $Acc \subseteq Access^T$,
    \item dependency matrix $DM \in \{-1,0,1\}^{T\times T}$;
  \end{itemize}
  \item alphabet $\Sigma:= T\times Access$ consists of pairs
  of thread identifiers and accesses;
  \item transition function $\delta_D$ is defined by:
  $\delta_D(q, \sigma) = q'$ where
  \begin{itemize}
    \item current state $q=(acc, dm)$
    \item action $\sigma=(i, a)$
    \item successor state $q'=(acc', dm')$ with
    \begin{itemize}
      \item $acc'=acc[i\mapsto a]$ 
      \item successor dependency matrix:
      \begin{align*}
       {dm'}_{a,b} & =
       \begin{cases}
       1                                                        & ;\  a=b=i\\
       -1                                                       & ;\ a=i, b\neq i\\
       dm_{p,q}                                                 & ;\ a\neq i, b\neq i \\
       0                                                        & ;\ a\neq i, b=i, dm_{b,b}=0\\
       \bigvee_{l=1..n} \left (dm_{a,l}=1 \wedge dep_{l,i} \right) & ;\ a\neq i, b=i, dm_{b,b}\neq 0\\
       \end{cases}
      \end{align*}
      where $dep_{l,i}= dep(acc(l), a)$
    \end{itemize}
  \end{itemize}
\end{itemize}
\end{definition}

% What do the values -1, 0, 1 mean?

% the dependency matrix describes the following mutually exclusive conditions:
% - value 0  ... the thread b has not been executed at all so far.
% - value 1  ... there is a conflict between current access of thread b, and most recent access of thread a.
% - value -1 ... there is no such conflict.


\begin{definition}[Partial-Order Automaton]
\label{def:rautomaton}
%
  Given the dependency-chain automaton $A_D=(Q_D,\Sigma,\delta_D,q_0,F)$,
  a reduced dependency-chain automaton can be defined by constraining the
  transition function of the dependency-chain automaton by the following
  guard predicate:
%
    \begin{itemize}
      \item $guard(q_0,\sigma,.)=1$
      \item $guard(q, (i,a), q')=\bigwedge_{j>i}  ( {q.dm}_{j,i}(k) \neq -1 \vee \bigvee_{l<i} {{q'}.dm}_{j,l}=1)$
    \end{itemize}
%
  such that the reduced transition function is defined by:
%
    \begin{align*}
      \delta_{RD}(q, \sigma) =
        \begin{cases}
          q' & ; \ guard(q,\sigma,q')=1\\
          \epsilon & ; \ guard(q,\sigma,q')=0
        \end{cases}
    \end{align*}
%
We call this reduced automaton \emph{partial-order automaton}.
\end{definition}

\begin{lemma}
  The traces of the reduced dependency-chain automaton consists of
  all Mazurkievicz traces.
\end{lemma}

\section{Subsumption Order}
%
The following result is new: we define a ordering over states of the reduced dependency-chain automaton,
and prove that it constitues a simulation relation.
%
The consequence of this result is that the simulation relation can be used for state-subsumption reduction
and quotienting of automata.

Subsumption order relates the access sets and dependency matrices of two states,
and is formally defined as follows.

\begin{definition}[Subsumption Order $\preceq_{sub}$]
Let $A_{RD}=(Q_D,\Sigma,\delta_D,q_0,F)$ be a partial-order automaton.
We define the following binary relation $\preceq_{sub} \subseteq Q_D \times Q_D$.
Let $q=(acc, dm)$ and $q'=(acc', dm')$ be two states,
we define
\begin{align*}
\forall q'\in Q:\quad \epsilon &\preceq q' \\
\forall q,q'\in Q:\quad q &\preceq q' \quad :\overset{def}{\Longleftrightarrow} \quad acc_R \subseteq acc'_R \wedge acc_W \subseteq acc'_W \wedge dm \sqsubseteq dm'
\end{align*}
where the ordering $\sqsubseteq$ over dependency matrices is defined as follows:
\begin{align*}
dm \sqsubseteq dm' \quad :\overset{def}{\Longleftrightarrow} \quad \left (dm_{i,j}={dm'}_{i,j}\right) \vee \left (dm_{i,j}=-1 \right) \ .
\end{align*}

\end{definition}

\begin{theorem}[Subsumption is a Simulation]
  Subsumption order is a simulation relation.
\end{theorem}

\begin{proof}
Assume that $q_1 \preceq q_2$. 
By Definition~\ref{}, we need to show that 
\mbox{$\forall r_1\in succ(q_1)\ \exists r_2\in succ(q_2): r_1 \preceq r_2$}.
Therefore, further assume that \mbox{$r_1\in succ(q_1)$}.
To establish our claim, we need to find a $r_2\in succ(q_2)$ such that $r_1 \preceq r_2$.
Since $r_1\in succ(q_1)$, there exists $\sigma=(i,a)$ such that $\delta(q_1,\sigma)=r_1$.

\underline{Claim:} State $r_2 := \delta(q_2,\sigma)$ fulfills $r_1 \preceq r_2$.

We prove this property in the subsequent Lemma~\ref{def:decision_matrix_monotonic}.
$\qed$
\end{proof}

\begin{lemma}[Update Monotonicity]
\begin{align*}
q_1 \preceq q_2\quad \Longrightarrow \quad \delta(q_1,\sigma) \preceq \delta(q_2,\sigma)
\end{align*}

\label{def:decision_matrix_monotonic}
\end{lemma}

\begin{proof}
  At this point, we need to exploit the specific properties of the transition
  function of the partial-order automaton. To this end,
  we denote $q_1=(acc_1,dm_1)$, $q_2=(acc_2, dm_2)$.
  Further, we denote \mbox{$r_1=(acc_1[i\mapsto a], dm'_1)$} and \mbox{$r_1=(acc_2[i\mapsto a], dm'_2)$}.

  \begin{itemize}
    \item Note that $acc_1[i\mapsto a] \subseteq acc_2[i\mapsto a]$, as replacement is monotonic.
    \item Finally, we need to establish that the update on the decision matrix is monotonic $dm'_1 \sqsubseteq dm'_2$.
  \end{itemize}


\underline{Case: $a\neq i, b=i$:}

  Suppose that $(dm_1)_{b,b}=0$, then $(dm_2)_{b,b}=0$, and thus $(dm'_1)_{a,b}=(dm'_2)_{a,b}=0$.
  Therefore, assume $(dm_1)_{b,b}\neq 0$.
  Then the term that defines $(dm'_1)_{a,b}$ is:
  \begin{align*}
  \bigvee_{l=1..n} \left ({(dm_1)}_{a,l}=1 \wedge {dep}_{l,i} \right)
  \end{align*}
  is either $-1$ or $1$.
  It is $1$ if there exists an index $l$ such that ${(dm_1)}_{a,l}=1$ and ${dep}_{l,i}$.
  If it is $1$, by the ordering constraint $dm_1 \sqsubseteq dm_2$, ${(dm_1)}_{a,l}=1$ implies
  ${(dm_2)}_{a,l}=1$. Altogether, therefore ${(dm'_1)}_{a,b}=1$ implies ${(dm'_2)}_{a,b}=1$.
$\qed$
\end{proof}

%
When applying a simulation relation for automata reduction,
it is desirable to have a largest simulation relation,
in order to obtain the maximally-achievable reduction.
%
The following lemma establishes this optimality property for subsumption order.
%
\begin{theorem}[Subsumption is the largest Simulation]

\end{theorem}
%
\begin{proof}


$\qed$
\end{proof}
%

The simulation result can be used for state subsumption in 
the automaton defined in Defintion~\ref{def:rautomaton}.


While subsumption order is the largest simulation relation for
a given dependency-chain automaton, 
one can do better when composing the dependency-chain automaton
with a concrete program. 
These optimisation require leveraging knowledge about the program
to predict presence or absence of certain future accesses.


\bibliographystyle{plain} %or alpha or splncs
\bibliography{db}

\ifx\versiont\longversion
\input{appendix}
\fi


\end{document}
